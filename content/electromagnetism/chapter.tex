\chapter{Electromagnetism}

\section{Electrostatics}

\begin{definition}[Electric charge] A scalar quantity that can describe an object. It can be positive or negative
\end{definition}

\begin{definition}[Elementary charge ($e$)] The smallest possible charge $e \approx 1.6 * 10^{-19} \, \Coulomb$.
\end{definition}

\subsection{Charging}

\begin{procedure}[Charging by friction] Rub two nonconductive objects together. The one with the greater electron affinity (electronegativity) becomes negatively charged.
\end{procedure}

\begin{procedure}[Polarization] When two objects near each other, their charge distributions will change in order to ensure that similar charges in the different objects don't get too near each other.
\end{procedure}

% TODO: Conduction and induction

\subsection{Electric fields}

\begin{definition}[Permittivity of free space ($\varepsilon_0$)]
  \[
    \varepsilon_0 \approx 8.85 * 10^{-12} \frac{\Coulomb^2}{\Newton \meter^2}
  \]
\end{definition}

\begin{definition}[Coulomb's constant ($k$)]
  \[
    k = \frac{1}{4\pi\varepsilon_0} \approx 8.99 * 10^9 \, \frac{\Newton \meter^2}{\Coulomb^2}
  \]
\end{definition}

\begin{namedlaw}[Coulomb's Law]
  If stationary point or spherical charges $q_1$ and $q_2$ are near each other but not touching, then the magnitude of the electrostatic force between the two objects is
  \[
    |F_e| = \frac{k |q_1| |q_2|}{r^2} = \frac{|q_1| |q_2|}{4 \pi \varepsilon_0 r^2}
  \]
\end{namedlaw}

\begin{definition}[Electric field]
  The vector field $E$ ($\frac{\Newton}{\Coulomb}$), which applies to all objects.

  The electric field points away from positive charges and towards negative charges.
\end{definition}

\begin{law}
  The electrostatic force on a stationary point or spherical charge $q$ is $F_e = Eq$, where $E$ is the strength and direction of the electric field.
\end{law}

\begin{law}[by Coulomb's Law]
  The electric field at distance $r$ due to a stationary point or spherical charge $Q$ is
  \[
    E = \frac{kQ}{r^2} = \frac{Q}{4\pi\varepsilon_0 r^2}
  \]

  The electric field at distance $r$ due to a charged object whose total charge is $\int dq$ is
  \[
    E = \int \frac{k \,dq}{r^2} = \int \frac{dq}{4 \pi \varepsilon_0 r^2}
  \]
  where $r$'s tail is at the charge and points towards the point for which we want to calculate $E$.
\end{law}

\begin{definition}[Charge density]
  \begin{align*}
    \text{3D: density} \qquad & \rho = \frac{Q}{v} = \frac{dq}{dV} & (\frac{\Coulomb}{\meter^3}) \\
    \text{2D: surface density} \qquad & \sigma = \frac{Q}{A} = \frac{dq}{dA} & (\frac{\Coulomb}{\meter^2}) \\
    \text{1D: linear density} \qquad & \lambda = \frac{Q}{L} = \frac{dq}{dx} & (\frac{\Coulomb}{\meter}) \\
  \end{align*}
\end{definition}

\begin{example}[E-field of a sheet of charge]
  The magnitude of the E-field caused by a sheet of charge is
  \[
    E = \frac{\sigma}{2\varepsilon_0}
  \]
  where the E-field points away from the sheet if the charge is positive, and towards the sheet if the charge is negative.
\end{example}

\begin{example}[E-field of a cylindrical charge]
  The magnitude of the E-field caused by a cylindrical charge of radius $R$ with negligible end effects, at radius $r$ from its axis, is:
  \[
    \frac{R\sigma}{r\varepsilon_0}
  \]
\end{example}

\subsection{Electric flux}

\begin{definition}[Electric flux ($\Phi$)]
  The flux through a surface with area vector $\vec{A}$ is
  \[
    \Phi = \int \vec{E} \dotp d\vec{A} \qquad \left(\frac{\Newton \meter^2}{\Coulomb}\right)
  \]
  where $\vec{E}$ is the electric field that passes through the surface, and the area vector's magnitude is the area and direction is normal/perpendicular to the surface.
\end{definition}

\begin{definition}[Net flux]
  The flux through any surface that encloses a charge.
\end{definition}

\begin{namedtheorem}[Gauss's Law]
  The net flux is equivalent to the enclosed charge divided by the permittivity of free space, and can be equated to the flux through the surface:
  \[
    \Phi_{\text{net}} = \frac{q}{\varepsilon_0} = 4\pi k q = \oint \vec{E} \dotp d\vec{A}
  \]
\end{namedtheorem}

\subsection{Electric potential}

\begin{definition}[Electric potential energy]
  The work required to move a charge from a reference position to its current location in the electric field.

  When the electrostatic force does work $W_e$ on the object, its electric potential energy decreases and its kinetic energy increases:
  \[
    \Delta U_e = -W_e
  \]
\end{definition}

\begin{definition}[Electric potential]
  The work per unit charge required to move a charge from a reference position to its current location in the electric field.
\end{definition}

\begin{definition}[Electric potential difference / Voltage]
  The difference in electric potential between two points.

  \[
    \Delta V = \frac{\Delta U_e}{q} = -\frac{W_e}{q} = -\int \vec{E}(\vec{r}) \dotp d\vec{r} 
  \]

  Therefore,
  \[
    \vec{E} = - \frac{dV}{dr}
  \]
\end{definition}

\begin{definition}[Equipotential line/surface]
  A line/surface where for each point on the line/surface, the electric potential is the same.
  
  Graphs of equipotential lines where each line's electric potential is an integer multiple of some number produce a topographic-like map of the electric field, where each of the equipotential lines can be viewed as contour lines.
\end{definition}

\begin{theorem}[Potential caused by a point or spherical charge]
  The electric potential outside a point or spherical charge, at distance $r$ from the center of the charge, is
  \[
    \frac{kq}{r}
  \]

  The electric potential inside a spherical charge of radius $R$ is
  \[
    \frac{kq}{R}
  \]

  (This derives from Coulomb's Law.)
\end{theorem}

\subsection{Capacitance}

\begin{definition}[Capacitor]
  At least one charged conductor, usually two.

  Common capacitor shapes:
  \begin{itemize}
    \item Parallel plate
    \item Cylindrical
    \item Spherical
  \end{itemize}
\end{definition}

\begin{definition}[Capacitance]
  Where $Q$ is the magnitude of the charge on one of the plates, and $V$ is the potential difference between two plates:

  \[
    C = \frac{Q}{V} \qquad \left(\Farad = \frac{\Coulomb}{\Volt}\right)
  \]

  The capacitance remains constant for a given capacitor regardless of any change in charge or voltage; it only depends on the shape of the capacitor.

  The capacitance is a positive (unsigned) value.
\end{definition}

\begin{example}[Parallel plate capacitor]
  The capacitance of a capacitor containing two parallel plates, where each plate has area $A$ and the distance between the plates is $d$, is
  \[
    C = \frac{\varepsilon_0 A}{d}
  \]
\end{example}

\begin{example}[Spherical shell capacitor]
  The capacitance of a capacitor containing two concentric spherical shells, where the smaller shell has radius $a$ and the larger has radius $b$, is
  \[
    C = \frac{ab}{bk - ak}
  \]
\end{example}

\begin{example}[Cylindrical capacitor]
  The capacitance of a capacitor containing two concentric cylindrical shells with negligible end effects, where the smaller shell has radius $a$ and the larger has radius $b$ and both have length $L$, is
  \[
    C = \frac{2 \pi \varepsilon_0 L}{\ln\left(\frac{b}{a}\right)}
  \]
\end{example}

\begin{definition}[Dielectric]
  An object that can be polarized.
\end{definition}

\begin{definition}[Permittivity]
  The permittivity of space containing a dielectric is equal to
  \[
    K\varepsilon_0
  \]
  where $K$ is the dielectric's \textbf{dielectric constant}
\end{definition}

\begin{definition}[Current ($i$)]
  The rate at which charges flow through a conductor.

  \[
    i = \frac{dq}{dt}
  \]
\end{definition}