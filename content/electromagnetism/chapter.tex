\chapter{Electromagnetism}

\section{Electrostatics}

\begin{definition}[Electric charge] A scalar quantity that can describe an object. It can be positive or negative
\end{definition}

\begin{definition}[Elementary charge ($e$)] The smallest possible charge $e \approx 1.6 * 10^{-19} \, \Coulomb$.
\end{definition}

\subsection{Charging}

\begin{procedure}[Charging by friction] Rub two nonconductive objects together. The one with the greater electron affinity (electronegativity) becomes negatively charged.
\end{procedure}

\begin{procedure}[Polarization] When two objects near each other, their charge distributions will change in order to ensure that similar charges in the different objects don't get too near each other.
\end{procedure}

% TODO: Conduction and induction

\subsection{Electric fields}

\begin{definition}[Permittivity of free space ($\varepsilon_0$)]
  \[
    \varepsilon_0 \approx 8.85 * 10^{-12} \frac{\Coulomb^2}{\Newton \meter^2}
  \]
\end{definition}

\begin{definition}[Coulomb's constant ($k$)]
  \[
    k = \frac{1}{4\pi\varepsilon_0} \approx 8.99 * 10^9 \, \frac{\Newton \meter^2}{\Coulomb^2}
  \]
\end{definition}

\begin{namedlaw}[Coulomb's Law]
  If stationary point or spherical charges $q_1$ and $q_2$ are near each other but not touching, then the magnitude of the electrostatic force between the two objects is
  \[
    |F_e| = \frac{k |q_1| |q_2|}{r^2} = \frac{|q_1| |q_2|}{4 \pi \varepsilon_0 r^2}
  \]
\end{namedlaw}

\begin{definition}[Electric field]
  The vector field $E$ ($\frac{\Newton}{\Coulomb}$), which applies to all objects.

  The electric field points away from positive charges and towards negative charges.
\end{definition}

\begin{law}
  The electrostatic force on a stationary point or spherical charge $q$ is $F_e = Eq$, where $E$ is the strength and direction of the electric field.
\end{law}

\begin{law}[by Coulomb's Law]
  The electric field at distance $r$ due to a stationary point or spherical charge $Q$ is
  \[
    E = \frac{kQ}{r^2} = \frac{Q}{4\pi\varepsilon_0 r^2}
  \]

  The electric field at distance $r$ due to a charged object whose total charge is $\int dq$ is
  \[
    E = \int \frac{k \,dq}{r^2} = \int \frac{dq}{4 \pi \varepsilon_0 r^2}
  \]
  where $r$'s tail is at the charge and points towards the point for which we want to calculate $E$.
\end{law}

\begin{definition}[Charge density]
  \begin{align*}
    \text{3D: density} \qquad & \rho = \frac{Q}{v} = \frac{dq}{dV} & (\frac{\Coulomb}{\meter^3}) \\
    \text{2D: surface density} \qquad & \sigma = \frac{Q}{A} = \frac{dq}{dA} & (\frac{\Coulomb}{\meter^2}) \\
    \text{1D: linear density} \qquad & \lambda = \frac{Q}{L} = \frac{dq}{dx} & (\frac{\Coulomb}{\meter}) \\
  \end{align*}
\end{definition}

\subsection{Electric flux}

\begin{definition}[Electric flux ($\Phi$)]
  The flux through a surface with area vector $\vec{A}$ is
  \[
    \Phi = \int \vec{E} \dotp d\vec{A} \qquad \left(\frac{\Newton \meter^2}{\Coulomb}\right)
  \]
  where $\vec{E}$ is the electric field that passes through the surface, and the area vector's magnitude is the area and direction is normal/perpendicular to the surface.
\end{definition}

\begin{definition}[Net flux]
  The flux through any surface that encloses a charge.
\end{definition}

\begin{namedtheorem}[Gauss's Law]
  The net flux is equivalent to the enclosed charge divided by the permittivity of free space, and can be equated to the flux through the surface:
  \[
    \Phi_{\text{net}} = \frac{q}{\varepsilon_0} = 4\pi k q = \oint \vec{E} \dotp d\vec{A}
  \]
\end{namedtheorem}

