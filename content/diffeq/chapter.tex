\chapter{Differential Equations}

A \textit{differential equation} is an equation involving a quantity and one or more of its derivatives.

\section*{1 Ordinary Differential Equations}

An ODE involves the derivative of the dependent variable with respect to a single independent variable.

\subsection{Solving by Integration}

\begin{definition}[Solution to a differential equation] A function $y = f(x)$ that satisfies the differential equation when $f$ and its derivatives are substituted into the equation.
\end{definition}

% \begin{procedure}[Making a direction field] This procedure can be used to approximately graph a differential equation, even if an explicit solution cannot be found.
  
%   Factor the equation in terms of $y' = f'(x)$. At each $x,y$-value on a grid of a graph, evaluate $y'$ and draw a short line with the slope $y'$ at the $x,y$-value.
% \end{procedure}

\begin{procedure}[Euler's Method] To numerically approximates the solution to the differential equation $y' = F(x, y)$ with $y(x_0) = y_0$,
  \[
    y_n = y_{n - 1} + F(x_{n-1}, y_{n - 1})(x_n - x_{n - 1})
  \]
\end{procedure}

\begin{definition}[Separable Equation] A separable equation is a differential equation where
  \[
    \frac{dy}{dx} = g(x)f(y)
  \]
  for some function $g(x)$ which depends only on $x$ and $f(y)$ which depends only on $y$.
\end{definition}

\begin{theorem} For a separable equation,
  \[
    \int \frac{1}{f(y)} dy = \int g(x) dx
  \]
\end{theorem}

\begin{definition}[Logistic Differential Equation] For a population $P$ which increases exponentially ($\frac{dP}{dt} \approx kP$) when the population is small compared to the carrying capacity $M$ but where the environment cannot sustain a population larger than $M$,
  \[
    \frac{dP}{dt} = kP \left(1 - \frac{P}{M}\right)
  \]
  Then
  \[
    P(t) = \frac{M}{1 + \left(\frac{M}{P_0} - 1\right)e^{-kt}}.
  \]
  and
  \[
    \frac{d^2P}{dt^2} = k^2P \left(1 - \frac{P}{M}\right) \left(1 - \frac{2P}{M}\right)
  \]
\end{definition}

\subsection{Initial-Value Problems}

\begin{definition}[Initial-Value problem]
    Assuming the function $f$ is continuous, then the function $y$ is a solution of the IVP given that
    \[
        \frac{dy}{dx} = f(x,y), y(x_0) = y_0,
    \]
    Where $x_0$ is called the \textit{initial point} for the IVP and $y_0$ is the \textit{initial value}.
\end{definition}

\subsection{Existence and Uniqueness of Solutions}

\begin{theorem}[Existence and Uniqueness]
    If $f$ is continuous, then the function $f$ as previously defined has at least one solution on the interval of continuity. If at least one solution exists and $\frac{\partial^{}f}{\partial^{}y}$ is continuous on the same interval, the solution is unique.
\end{theorem}