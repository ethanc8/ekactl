
\subsection{Exact Equations}

An exact equation is a differential equation in the form
\[
    M(x, y) + N(x, y)\frac{dy}{dx} = 0
\]
where $M, N$ are continuously differentiable. They also represent partial derivatives of the potential function $\psi$

\[
    \psi_x + \psi_y\frac{dy}{dx} = 0
\]

where $\psi(x, y)$ is an existing \textit{potential function}, as seen in Multivariable Calculus. The equation is exact only if,
\begin{itemize}
    \item $\psi_{xy} = \psi_{yx}$
    \item Domain of functions above is open and topologically simply-connected (continuous)
\end{itemize}

So if the potential function exists, then the solution to the equation would be
\[
    \psi(x, y) = 0
\]