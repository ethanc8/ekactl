\chapter{Mathematics}

\section{Notation}

$\deg~p(x)$ means the degree of polynomial $p$.

$\text{LC}~p(x)$ means the leading coefficient of polynomial $p$.

\section{Rational functions}

For a rational function $f(x) = \frac{p(x)}{q(x)}$, cancel out any common factors, then:

\begin{itemize}
	\item When $\deg~p(x) = \deg~q(x)$:
	\begin{itemize}
		\item HA: $y = \frac{\text{LC}~p(x)}{\text{LC}~q(x)}$
		\item VA: roots of $q(x)$
	\end{itemize}
	\item When $\deg~p(x) < \deg~q(x)$:
	\begin{itemize}
		\item HA: $y = 0$
		\item x-intercept: roots of $p(x)$
		\item VA: roots of $q(x)$
	\end{itemize}
	\item When $\deg~p(x) > \deg~q(x)$:
	\begin{itemize}
		\item HA: none
		\item slant asymptote: $\frac{p(x)}{q(x)}$ excluding remainder
		\item VA: roots of $q(x)$
	\end{itemize}
\end{itemize}

\section{Polynomials}

\subsection{Linear equations}

\[\begin{aligned}
	\text{Slope-intercept form:}&~y = mx + b \\
	\text{Point-slope form:}&~y - y_1 = m(x - x_1)~\text{for point $(x, y)$} \\
	\text{Standard form:}&~ax + by = c
\end{aligned}\]


\subsection{Quadratic equations}

\[\begin{aligned}
	\text{Standard form:}&~y = ax^2 + bx + c \\
	\text{Vertex form:}&~y = a(x-h)^2 + k~\text{for vertex $(h, k)$} \\
	\text{Sum of roots:}&~\frac{-b}{a} \\
	\text{Product of roots:}&~\frac{c}{a} \\
\end{aligned}\]

\subsection{Higher-degree polynomials}
In a polynomial
\[
	a_n x^n + a_{n-1} x^{n-1} + \dotsb + a_1 x + a_0 = 0
\]
, with roots
\[
	r_1, r_2, r_3, \dotsc , r_n
\]
then:
\[\begin{aligned}
	r_1 + r_2 + r_3 + \dotsb + r_n = \sum_{k=1}{n} r_k = -\frac{a_{n-1}}{a_n}
\end{aligned}\]

\columnbreak

\section{Sequences and Series}

\subsection{Explicit formulas}
\[\begin{aligned}
	\text{Aritmetic sequence:}~a_n &= a_1 + r(n - 1) \\
	\text{Geometric sequence:}~a_n &= a_1 * r^{n - 1} \\
	\text{Harmonic sequence:}~a_n &= \frac{1}{a_1 + r(n - 1)}
\end{aligned}\]

\subsection{Arithmetic and Geometric Series}

\[\begin{aligned}
	\sum_{j=1}^{n} (a_1 + r(n-1)) &= \frac{n}{2}(2a_1 + (n - 1) d) \\
	\sum_{j=1}^{n} (a_1 * r^{n - 1}) &= \frac{a_1 (1-r^n)}{1-r} \\
	\sum_{j=1}^{\infty} (a_1 * r^{n - 1}) &= \frac{a_1}{1-r} ~\text{for}~ r \in [-1, 1] \\
\end{aligned}\]

\subsection{Special Sums}

\[
\begin{aligned}
	\sum_{j=1}^{n} c &= nc \\
	\sum_{j=1}^{n} ca_j &= c \sum_{j=1}^{n} a_j \\
	\sum_{j=1}^{n} (a_j + b_j) &= \sum_{j=1}^{n} a_j + \sum_{j=1}^{n} b_j \\
	\sum_{j=1}^{n} j &= \frac{n(n+1)}{2} \\
	\sum_{j=1}^{n} j^2 &= \frac{n(n+\frac{1}{2})(n+1)}{3} = \frac{n(2n+1)(n+1)}{6} \\
	\sum_{j=1}^{n} j^3 &= \frac{n^2(n+1)^2}{4} \\
	\sum_{j=1}^{n} j^4 &= \frac{n(n+1)(2n+1)(3n^2 + 3n - 1)}{30} \\
\end{aligned}
\]

\columnbreak

\section{Trigonometry}

\newcolumntype{L}{>{$}l<{$}} % math-mode version of "l" column type

\begin{center}
\begin{tabular}{L L | L L L} 
	% \hline
	\degree & \text{rad} & \sin & \cos & \tan \\ 
	\hline
	0\degree & 0 & 0 & 1 & 0 \\
	30\degree & \frac{\pi}{6} & \frac{1}{2} & \frac{\sqrt{3}}{2} & \frac{1}{\sqrt{3}} \\
	45\degree & \frac{\pi}{4} & \frac{\sqrt{2}}{2} & \frac{\sqrt{2}}{2} & 1 \\
	60\degree & \frac{\pi}{3} & \frac{\sqrt{3}}{2} & \frac{1}{2} & \sqrt{3} \\
	90\degree & \frac{\pi}{2} & 1 & 0 & undef \\
	% \hline
\end{tabular}
\end{center}

\subsection{Double-Angle and Related Identities}

% https://math.libretexts.org/Courses/Reedley_College/Trigonometry/03%3A_Trigonometric_Identities_and_Equations/3.04%3A_Sum-to-Product_and_Product-to-Sum_Formulas
\subsubsection{Product-to-Sum Formulas}
\[\begin{aligned}
	\cos \alpha \cos \beta &= \dfrac{1}{2}[\cos(\alpha-\beta)+\cos(\alpha+\beta)] \\
	\sin \alpha \cos \beta &= \dfrac{1}{2}[\sin(\alpha+\beta)+\sin(\alpha-\beta)] \\
	\sin \alpha \sin \beta &= \dfrac{1}{2}[\cos(\alpha-\beta)-\cos(\alpha+\beta)] \\
	\cos \alpha \sin \beta &= \dfrac{1}{2}[\sin(\alpha+\beta)-\sin(\alpha-\beta)] \\
\end{aligned}\]

\subsubsection{Sum-to-Product Formulas}
\[\begin{aligned}
	\sin \alpha+\sin \beta &= 2\sin(\dfrac{\alpha+\beta}{2})\cos(\dfrac{\alpha-\beta}{2}) \nonumber \\
	\sin \alpha-\sin \beta &= 2\sin(\dfrac{\alpha-\beta}{2})\cos(\dfrac{\alpha+\beta}{2}) \nonumber \\
	\cos \alpha-\cos \beta &= -2\sin(\dfrac{\alpha+\beta}{2})\sin(\dfrac{\alpha-\beta}{2}) \nonumber \\
	\cos \alpha+\cos \beta &= 2\cos(\dfrac{\alpha+\beta}{2})\cos(\dfrac{\alpha-\beta}{2}) \nonumber \\
\end{aligned}\]

\columnbreak

\section{Geometric Transformations}

\subsection{Vertex matrices}

A \textbf{vertex matrix} is a matrix in which the columns represent points in a shape and the rows represent the components.

For example, the triangle $\triangle ABC$, with points $A(-8, 7)$, $B(-4, 10)$, and $C(-1, -3)$ is represented by the following vertex matrix:
\[\begin{bNiceMatrix}[first-col, first-row]
	  & A  & B  & C \\
	x & -8 & -4 & -1 \\
	y & 7  & 10 & -3
\end{bNiceMatrix}\]

\subsection{Translations}

To translate a figure $h$ units right and $k$ units up, add an appropriate matrix.

For example, to translate $\triangle ABC$:
\[\begin{bNiceMatrix}
  	A_x & B_x & C_x \\
  	A_y & B_y & C_y
\end{bNiceMatrix} + \begin{bNiceMatrix}
	h & h & h \\
	k & k & k
\end{bNiceMatrix}\]

To translate $f(x)$:
\[
	f'(x) = f(x-h) + k
\]

\subsection{Dilations}

To dilate a figure by a factor of $k$, multiply the figure's vertex matrix by $k$.

For example, to dilate $\triangle ABC$:
\[
	k * 
	\begin{bNiceMatrix}
  		A_x & B_x & C_x \\
  		A_y & B_y & C_y
	\end{bNiceMatrix}
\]

To dilate a function by a factor of $m$ in the $x$-direction and $n$ in the $y$-direction:
\[
	f'(x) = n * f(\frac{x}{m})
\]

\subsection{Reflections}

\subsubsection{Over the $x$-axis}

