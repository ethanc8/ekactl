% Written by Anders Sjoqvist and Ulf Lundstrom, 2009
% The main sources are: tinyKACTL, Beta and Wikipedia

\chapter{Mathematics}

\section{Notation}

$\deg~p(x)$ means the degree of polynomial $p$.

$\text{LC}~p(x)$ means the leading coefficient of polynomial $p$.

\section{Rational functions}

For a rational function $f(x) = \frac{p(x)}{q(x)}$, cancel out any common factors, then:

\begin{itemize}
	\item When $\deg~p(x) = \deg~q(x)$:
	\begin{itemize}
		\item HA: $y = \frac{\text{LC}~p(x)}{\text{LC}~q(x)}$
		\item VA: roots of $q(x)$
	\end{itemize}
	\item When $\deg~p(x) < \deg~q(x)$:
	\begin{itemize}
		\item HA: $y = 0$
		\item x-intercept: roots of $p(x)$
		\item VA: roots of $q(x)$
	\end{itemize}
	\item When $\deg~p(x) > \deg~q(x)$:
	\begin{itemize}
		\item HA: none
		\item slant asymptote: $\frac{p(x)}{q(x)}$ excluding remainder
		\item VA: roots of $q(x)$
	\end{itemize}
\end{itemize}

\section{Common functions}

\subsection{Linear equations}

\[\begin{aligned}
	\text{Slope-intercept form:}&~y = mx + b \\
	\text{Point-slope form:}&~y - y_1 = m(x - x_1)~\text{for point $(x, y)$} \\
	\text{Standard form:}&~ax + by = c
\end{aligned}\]


\subsection{Quadratic equations}

\[\begin{aligned}
	\text{Standard form:}&~y = ax^2 + bx + c \\
	\text{Vertex form:}&~y = a(x-h)^2 + k~\text{for vertex $(h, k)$}
\end{aligned}\]

\section{Sequences and Series}

\subsection{Explicit formulas}
\[\begin{aligned}
	\text{Aritmetic sequence:}~a_n &= a_1 + r(n - 1) \\
	\text{Geometric sequence:}~a_n &= a_1 * r^{n - 1} \\
	\text{Harmonic sequence:}~a_n &= \frac{1}{a_1 + r(n - 1)}
\end{aligned}\]

\subsection{Arithmetic and Geometric Series}

\[\begin{aligned}
	\sum_{j=1}^{n} (a_1 + r(n-1)) &= \frac{n}{2}(2a_1 + (n - 1) d) \\
	\sum_{j=1}^{n} (a_1 * r^{n - 1}) &= \frac{a_1 (1-r^n)}{1-r} \\
	\sum_{j=1}^{\infty} (a_1 * r^{n - 1}) &= \frac{a_1}{1-r} ~\text{for}~ r \in [-1, 1] \\
\end{aligned}\]

\subsection{Special Sums}

\[
\begin{aligned}
	\sum_{j=1}^{n} c &= nc \\
	\sum_{j=1}^{n} ca_j &= c \sum_{j=1}^{n} a_j \\
	\sum_{j=1}^{n} (a_j + b_j) &= \sum_{j=1}^{n} a_j + \sum_{j=1}^{n} b_j \\
	\sum_{j=1}^{n} j &= \frac{n(n+1)}{2} \\
	\sum_{j=1}^{n} j^2 &= \frac{n(n+\frac{1}{2})(n+1)}{3} = \frac{n(2n+1)(n+1)}{6} \\
	\sum_{j=1}^{n} j^3 &= \frac{n^2(n+1)^2}{4} \\
	\sum_{j=1}^{n} j^4 &= \frac{n(n+1)(2n+1)(3n^2 + 3n - 1)}{30} \\
\end{aligned}
\]

