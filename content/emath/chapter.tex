\chapter{Mathematics}

\section{Logarithms}

\[\begin{aligned}
	\log_b(MN)&=\log_b(M)+\log_b(N) \\
	\log_b\left(\dfrac{M}{N}\right)&=\log_b(M)-\log_b(N) \\
	\log_b(M^p)&=p\cdot\log_b(M) \\
	\log_{b}(a)&=\dfrac{\log_{x}(a)}{\log_{x}(b)} \\
	\log_b(b)&=1 \\
\end{aligned}\]

\section{Notation}

$\deg~p(x)$ means the degree of polynomial $p$.

$\text{LC}~p(x)$ means the leading coefficient of polynomial $p$.

\section{Rational functions}

For a rational function $f(x) = \frac{p(x)}{q(x)}$, cancel out any common factors, then:

\begin{itemize}
	\item For all rational functions:
	\begin{itemize}
		\item VA: roots of $q(x)$
		\item Roots: roots of $p(x)$
	\end{itemize}
	\item When $\deg~p(x) = \deg~q(x)$:
	\begin{itemize}
		\item HA: $y = \frac{\text{LC}~p(x)}{\text{LC}~q(x)}$
	\end{itemize}
	\item When $\deg~p(x) < \deg~q(x)$:
	\begin{itemize}
		\item HA: $y = 0$
	\end{itemize}
	\item When $\deg~p(x) > \deg~q(x)$:
	\begin{itemize}
		\item HA: none
		\item slant asymptote: $\frac{p(x)}{q(x)}$ excluding remainder
	\end{itemize}
\end{itemize}

\section{Polynomials}

\subsection{Linear equations}

\[\begin{aligned}
	\text{Slope-intercept form:}&~y = mx + b \\
	\text{Point-slope form:}&~y - y_1 = m(x - x_1)~\text{for point $(x, y)$} \\
	\text{Standard form:}&~ax + by = c
\end{aligned}\]


\subsection{Quadratic equations}

\[\begin{aligned}
	\text{Standard form:}&~y = ax^2 + bx + c \\
	\text{Vertex form:}&~y = a(x-h)^2 + k~\text{for vertex $(h, k)$} \\
	\text{Sum of roots:}&~\frac{-b}{a} \\
	\text{Product of roots:}&~\frac{c}{a} \\
\end{aligned}\]

\subsection{Higher-degree polynomials}
In a polynomial
\[
	a_n x^n + a_{n-1} x^{n-1} + \dotsb + a_1 x + a_0 = 0
\]
, with roots
\[
	r_1, r_2, r_3, \dotsc , r_n
\]
then:
\[\begin{aligned}
	r_1 + r_2 + r_3 + \dotsb + r_n = \sum_{k=1}{n} r_k = -\frac{a_{n-1}}{a_n}
\end{aligned}\]

\columnbreak


\section{Sequences and Series}

\subsection{Explicit formulas}
\[\begin{aligned}
	\text{Aritmetic sequence:}~a_n &= a_1 + r(n - 1) \\
	\text{Geometric sequence:}~a_n &= a_1 * r^{n - 1} \\
	\text{Harmonic sequence:}~a_n &= \frac{1}{a_1 + r(n - 1)}
\end{aligned}\]

\subsection{Arithmetic and Geometric Series}

In the following equations, substituting $j=1$ with $j=0$, $j-1$ with $j$, and $a_1$ with $a_0$ will produce the same result.

\[\begin{aligned}
	\sum_{j=1}^{n} (a_1 + r(j-1)) &= \frac{n}{2}(2a_1 + (n - 1) d) \\
	\sum_{j=1}^{n} (a_1 * r^{j - 1}) &= \frac{a_1 (1-r^n)}{1-r} \\
	\sum_{j=1}^{\infty} (a_1 * r^{j - 1}) &= \frac{a_1}{1-r} ~\text{for}~ r \in [-1, 1] \\
\end{aligned}\]

\subsection{Special Sums}

\[
\begin{aligned}
	\sum_{j=1}^{n} c &= nc &
	\sum_{j=1}^{n} ca_j &= c \sum_{j=1}^{n} a_j \\
	\sum_{j=1}^{n} (a_j + b_j) &= \sum_{j=1}^{n} a_j + \sum_{j=1}^{n} b_j &
	\sum_{j=1}^{n} j &= \frac{n(n+1)}{2} \\
	\sum_{j=1}^{n} j^2 &= \frac{n(n+\frac{1}{2})(n+1)}{3} &
	% = \frac{n(2n+1)(n+1)}{6} \\
	\sum_{j=1}^{n} j^3 &= \frac{n^2(n+1)^2}{4} \\
	% \sum_{j=1}^{n} j^4 &= \frac{n(n+1)(2n+1)(3n^2 + 3n - 1)}{30} \\
	                   &= \frac{n(2n+1)(n+1)}{6} \\
\end{aligned}
\]

% \columnbreak

\section{Trigonometry}

\newcolumntype{L}{>{$}l<{$}} % math-mode version of "l" column type

\begin{center}
\begin{tabular}{L L | L L L} 
	% \hline
	\degree & \text{rad} & \sin & \cos & \tan \\ 
	\hline
	0\degree & 0 & 0 & 1 & 0 \\
	30\degree & \frac{\pi}{6} & \frac{1}{2} & \frac{\sqrt{3}}{2} & \frac{1}{\sqrt{3}} \\
	45\degree & \frac{\pi}{4} & \frac{\sqrt{2}}{2} & \frac{\sqrt{2}}{2} & 1 \\
	60\degree & \frac{\pi}{3} & \frac{\sqrt{3}}{2} & \frac{1}{2} & \sqrt{3} \\
	90\degree & \frac{\pi}{2} & 1 & 0 & \text{undef} \\
	% \hline
\end{tabular}
\end{center}

\subsection{Law of Sines and Cosines}
\[
	\frac{\sin(A)}{a} = \frac{\sin(B)}{b} = \frac{\sin(C)}{c} \qquad
	c^2 = a^2 + b^2 - 2ab\cos(C)
\]
\subsection{Triangle area}
\[
	K = \frac{1}{2}bh \qquad
	K = \frac{1}{2}bc \sin(A) \qquad
	K = \sqrt{s(s-a)(s-b)(s-c)}
\]
\subsection{More identities}
\[\begin{aligned}
	(\sin A)^2 + (\cos A)^2 &= 1 &
	(\tan A)^2 + 1 &= (\sec A)^2 \\
	\sin(\frac{\pi}{2} - x) &= \cos(x) &
	(\cot A)^2 + 1 &= (\csc A)^2 \\
\end{aligned}\]\[
	\cos(-x) = \cos(x) \qquad
	\sin(-x) = \sin(x) \qquad
	\tan(-x) = \tan(x)
\]
\subsection{Slope}
Where $\alpha$ is the angle between the line and the x-axis, and $m$ is the slope of the line:
\[
	m = \tan \alpha
\]
\subsection{Sum and difference formulas}
\[\begin{aligned}
	\sin(A + B) &= \sin(A)\cos(B) + \cos(A)\sin(B) \\
	\sin(A - B) &= \sin(A)\cos(B) - \cos(A)\sin(B) \\
	\cos(A + B) &= \cos(A)\cos(B) - \sin(A)\sin(B) \\
	\cos(A + B) &= \cos(A)\cos(B) - \sin(A)\sin(B) \\
	\tan(A + B) &= \frac{\tan(A) + \tan(B)}{1-\tan(A)\tan(B)} \\
	\tan(A - B) &= \frac{\tan(A) - \tan(B)}{1+\tan(A)\tan(B)} \\
\end{aligned}\]\[\begin{aligned}
	\sin(2A) &= 2\sin(A)\cos(A) \\
	\cos(2A) &= (\cos A)^2 - (\sin A)^2 = 2(\cos A)^2 - 1 = 1 - 2(\sin A)^2 \\
	\tan(2A) &= \frac{2\tan(A)}{1 - (\tan A)^2}
\end{aligned}\]

\section{Vectors}
\newcommand{\dotp}{\boldsymbol{\cdot}}
\newcommand{\crossp}{\times}
\newcommand{\proj}{\text{proj}}
% \subsection{Vector arithmetic}

\[\begin{aligned}
	\vec{v} + \vec{w} &= \begin{bmatrix}
		v_x + w_x \\
		v_y + w_y \\
		v_z + w_z \\
	\end{bmatrix} \qquad
	c * \vec{v} = \begin{bmatrix}
		c * v_x \\
		c * v_y \\
		c * v_z \\
	\end{bmatrix} \\
	\vec{v} \dotp \vec{w} &= v_x w_x + v_y w_y + v_z w_z = |\vec{v}||\vec{w}| \cos(\theta) \\
	|\vec{v} \crossp \vec{w}| &= |\vec{v}||\vec{w}| \sin(\theta) = \text{area of parallelogram}\\
	\vec{v} \crossp \vec{w} &= \begin{vmatrix}
		\hat{i} & \hat{j} & \hat{k} \\
		v_x & v_y & v_z \\
		w_x & w_y & w_z
	\end{vmatrix} \qquad \vec{v} \crossp \vec{w} \perp \vec{v} \qquad \vec{v} \crossp \vec{w} \perp \vec{w} \\
	\vec{v} \perp \vec{w} &\iff \vec{v} \crossp \vec{w} = \vec{0} \qquad \vec{v} \parallel \vec{w} \iff \vec{v} \dotp \vec{w} = 0 \\
	\hat{v} &= \frac{\vec{v}}{|\vec{v}|} \qquad
	\text{proj}_{\vec{b}} \vec{v} = \frac{\vec{v} \dotp \vec{b}}{\vec{b} \dotp \vec{b}} * \vec{b} = (|\vec{v}| \cos(\theta))
\end{aligned}\]
\subsubsection{Right-hand rule}
To determine the direction of $\vec{v} \crossp \vec{w}$, put the side of the right hand on $\vec{v}$ and curl the fingers toward $\vec{w}$. The direction the thumb is pointing is the direction of $\vec{v} \crossp \vec{w}$.
\section{Polar}
\subsection{Polar and Cartesian sytems}
With point $(x, y) = (r; \theta) = (r; \beta)$, where $\theta$ is CCW from the $x$-axis and $\beta$ is a bearing, CW from the $y$-axis:
\[\begin{aligned}
	x &= r \cos(\theta) = r \sin(\beta) & y &= r \sin(\theta) = r \cos(\beta) \\
	r &= \sqrt{x^2 + y^2} & \theta &\equiv \arctan(\frac{y}{x}) \quad \beta \equiv \arctan(\frac{x}{y})
\end{aligned}\]
\subsection{Converting functions}

Try these substitutions in order:
\[
	x^2 = x^2 + y^2 \qquad \tan\theta = \frac{y}{x} \qquad
	x = r\cos\theta \qquad y = r\sin\theta
\]
\subsection{Limaçons and Petals}

The function $y = A \cos(B(\theta + C)) + D$ is equivalent to $y = A \cos(B\theta) + D$ rotated $C$ degrees/radians clockwise.

When $C$ is 0 and $B$ is 1, the x-intercepts are $A \pm D$ and the y-intercepts are $\pm D$, and it forms a limaçon.

When $C$ is 0, but $B \neq 1$, then this sometimes still holds. The x-intercepts may also be $\pm A \pm D$.

There are $B$ petals, with the axis of the first petal on the positive x-axis.

When $B$ is even and $|D| < 1$, then the number of petals is $2B$.

Using $\sin$ instead of $\cos$, limaçons have their axes on the positive y-axis, while for petals, the first petal starts from the positive x-axis and curves upwards.

\section{Complex}

\[
	\cis(\theta) = e^{i\theta} = \cos(\theta) + i \sin(\theta)
\]

To find the $n^{\text{th}}$ root of $x_r \cis(x_{\theta})$, solve the equation $z_r^n \cis(nz_{\theta}) = x_r \cis(x_{\theta} + 360\degree k)$ for $k \in \Real$.

\section{Function domain}

\begin{center}
	\begin{tabular}{L L L} 
		% \hline
		\text{Function} & \text{Domain $x$} & \text{Range $y$} \\ 
		\hline
		\log(x) & (0, \infty) & \Real \\
		\sqrt{x} & [0, \infty) & [0, \infty) \\
		\arcsin(x) & [-1, 1] & [-\frac{\pi}{2}, \frac{\pi}{2}] \\
		\arccos(x) & [-1, 1] & [0, \pi] \\
		\arctan(x) & \Real & (-\frac{\pi}{2}, \frac{\pi}{2}) \\
		% \hline
	\end{tabular}
\end{center}