\chapter{Calculus Theorems}

\newtheoremstyle{ethantheorem}%                % Name
  {}%                                     % Space above
  {}%                                     % Space below
  {}%                                     % Body font
  {}%                                     % Indent amount
  {\bfseries}%                            % Theorem head font
  {.}%                                    % Punctuation after theorem head
  { }%                                    % Space after theorem head, ' ', or \newline
  {\thmname{#1}\thmnumber{ #2}\thmnote{ (#3)}}%    

  \newtheoremstyle{ethandefinition}%                % Name
  {}%                                     % Space above
  {}%                                     % Space below
  {}%                                     % Body font
  {}%                                     % Indent amount
  {}%                            % Theorem head font
  {.}%                                    % Punctuation after theorem head
  { }%                                    % Space after theorem head, ' ', or \newline
  {\thmname{\textit{#1}}\thmnumber{ #2}\thmnote{\textit{ of }\textbf{#3}}}%    

\theoremstyle{ethantheorem}
\newtheorem*{theorem}{Theorem}

\theoremstyle{ethandefinition}
\newtheorem*{definition}{Definition}

\subsection*{1 Completeness}
\subsubsection*{1.3 Completeness}

\begin{theorem}[Completeness of the Real Numbers]
    Every nonempty subset $S$ of $\Real$ which is bounded above has a least upper bound $\sup S$.
\end{theorem}

\begin{definition}[Supremum ($\sup S$)]
    A number.
\end{definition}

\subsection*{2 Limits}

\subsubsection*{2.12 Squeeze Theorem}

\begin{theorem}[Squeeze Theorem]
  Let $f$ , $g$, and $h$ be defined for all $x \neq a$ over an open interval containing $a$. If
  $$
    f (x) \leq g(x) \leq h(x)
  $$
  for all $x \neq a$ in an open interval containing $a$ and
  $$
  \lim_{x \to a} f (x) = L = \lim_{x \to a} h(x)
  $$
  where $L \in R$, then $\lim_{x \to a} g(x) = L$.
\end{theorem}

\subsection*{3 Continuity}

\begin{definition}[Continuity at a point]
  Function $f$ is continuous at point $a$ if $\lim_{x \to a} f (x) = f (a)$.
\end{definition}

\begin{definition}
  $f$ has a \textbf{removable discontinuity} if $\lim_{x \to a} f (x) = L \in \Real$ (in this case either $f (a)$ is undefined, or $f (a)$ is defined by $L \neq f (a)$).
\end{definition}
\begin{definition}
  $f$ has a \textbf{jump discontinuity} if $\lim_{x \to a^-} f (x) = L_1 \in \Real$ and $\lim_{x \to a^+} f (x) = L_2 \in \Real$ but $L1 \neq L2$.
\end{definition}
\begin{definition}
  $f$ has an \textbf{infinite discontinuity} at $a$ if $\lim_{x \to a^-} f (x) = \pm\infty$ or $\lim_{x \to a^+} f (x) = \pm\infty$
\end{definition}

\begin{theorem}[Intermediate Value Theorem]
  If $f$ is continuous on $[a, b]$, then for any real number $L$ between $f (a)$ and $f (b)$ there exists at least one $c \in [a, b]$ such that $f (c) = L$. In other
  words, if $f$ is continuous on $[a, b]$, then the graph must cross the horizontal line $y = L$ at least once
  between the vertical lines $x = a$ and $x = b$.
\end{theorem}

\subsection*{4 Derivatives}