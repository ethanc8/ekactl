\chapter{Calculus Theorems}

\newtheoremstyle{ethantheorem}%                % Name
  {}%                                     % Space above
  {}%                                     % Space below
  {}%                                     % Body font
  {}%                                     % Indent amount
  {\bfseries}%                            % Theorem head font
  {.}%                                    % Punctuation after theorem head
  { }%                                    % Space after theorem head, ' ', or \newline
  {\thmname{#1}\thmnumber{ #2}\thmnote{ (#3)}}%    

\newtheoremstyle{ethannamedtheorem}%                % Name
  {}%                                     % Space above
  {}%                                     % Space below
  {}%                                     % Body font
  {}%                                     % Indent amount
  {\bfseries}%                            % Theorem head font
  {.}%                                    % Punctuation after theorem head
  { }%                                    % Space after theorem head, ' ', or \newline
  {\thmnumber{#2 }\thmnote{#3}}%   

\newtheoremstyle{ethandefinition}%                % Name
  {}%                                     % Space above
  {}%                                     % Space below
  {}%                                     % Body font
  {}%                                     % Indent amount
  {}%                            % Theorem head font
  {.}%                                    % Punctuation after theorem head
  { }%                                    % Space after theorem head, ' ', or \newline
  {\thmname{\textit{#1}}\thmnumber{ #2}\thmnote{\textit{ of }\textbf{#3}}}%    

\theoremstyle{ethantheorem}
\newtheorem*{theorem}{Theorem}

\theoremstyle{ethannamedtheorem}
\newtheorem*{namedtheorem}{Theorem}

\theoremstyle{ethandefinition}
\newtheorem*{definition}{Definition}

\subsection*{1 Completeness}
\subsubsection*{1.3 Completeness}

\begin{theorem}[Completeness of the Real Numbers]
  Every nonempty subset $S$ of $\Real$ which is bounded above has a least upper bound $\sup S$.
\end{theorem}

\begin{definition}[Supremum ($\sup S$)]
  A number such that 
  \begin{enumerate}[(1)]
    \item $s \leq \sup S$ for every $s \in S$ (which just says that $\sup S$ is an upper bound for $S$)
    \item If $u$ is any upper bound for S, then $\sup S \leq u$ (which says that $\sup S$ is the least upper bound for $S$).
  \end{enumerate}
\end{definition}

\begin{definition}[Infimum ($\inf S$)]
  A number such that 
  \begin{enumerate}[(1)]
    \item $\inf S \leq s$ for every $s \in S$ (i.e. $\inf S$ is an lower bound for $S$)
    \item If $l$ is any upper bound for $S$, then $l \leq \inf S$ (i.e. $\inf S$ is the greatest lower bound for $S$).
  \end{enumerate}
\end{definition}

\begin{theorem}
  Every nonempty subset $S$ of $\Real$ which is bounded below has a greatest lower bound.
\end{theorem}

\begin{theorem}
  If $\min S$ exists, then $\min S = \inf S$.
\end{theorem}

\begin{theorem}
  If $A \subset R$ and $c \geq 0$, and $cA := {ca : a \in A}$, $\sup cA = c \sup A$.
\end{theorem}

\subsubsection*{1.4 Consequences of Completeness}

\begin{theorem}[Rationals between Reals]
  For every two real numbers $a$ and $b$ with $a < b$, there exists a rational number $r$ satisfying $a < r < b$.
\end{theorem}

\subsubsection*{1.5 Nested Intervals Theorem}

\begin{namedtheorem}[Nested Intervals Theorem] \leavevmode \\
  If $I_n = [a_n, b_n] = \{x \in R : a_n \leq x \leq b_n\}$ s.t. $a_n \leq a_{n + 1}$ and $b_{n+1} \leq b_n$ for $n \in \Natural$, so that $I_1 \subseteq I_2 \subseteq I_3 \subseteq I_4 \subseteq \ldots$, then $\displaystyle \bigcap_{n=1}^{\infty}I_n \neq \emptyset$.

  If $\inf \{b_n - a_n\} = 0$, then $\displaystyle \bigcap_{n=1}^{\infty}I_n \{x\}$, where $x = \sup \{a_n\} = inf \{ b_n \}$.
\end{namedtheorem}

\subsubsection*{1.6 Capture Theorem}

\begin{namedtheorem}[Capture Theorem] If $A$ is a nonempty subset of $\Real$, then:
  \begin{enumerate}[(i)]
    \item If $A$ is bounded above, then any open interval containing $\sup A$ contains an element of $A$.
    \item Similarly, if $A$ is bounded below, then any open interval containing $\inf A$ contains an element of $A$.
  \end{enumerate}
\end{namedtheorem}

\subsubsection*{1.7 Binary Search}

If we binary-search for $x$ over $I_1 = [a_1, b_1]$ for $a_1, b_1 \in \Rational$, 
we define $I_n$ s.t. either $I_n := [a_{n-1}, \frac{a_{n-1} + b_{n-1}}{2}]$ or 
$I_n := [\frac{a_{n-1} + b_{n-1}}{2}, a_{n+1}]$, and we define $a_n := \inf I_n$ 
and $b_n := \sup I_n$. We define $A$ to be the set of all $a_n$, and $B$ to be 
the set of all $b_n$. 

Then, the size of $I_n = \frac{b_1 - a_1}{2^n} = b_n - a_n$, and $\displaystyle \bigcap_{n=1}^{\infty}I_n \{x\}$, where $x = \sup \{a_n\} = inf \{ b_n \}$.

\subsection*{2 Limits}

\subsubsection*{2.4 $\varepsilon$-$\delta$ definition of a Limit}

\begin{definition}[Limit]
  If $\displaystyle\lim_{x \to a} f(x) = L$, then for any $\varepsilon > 0$, there exists $\delta > 0$ s.t. for any $x \in (a - \delta, a) \cup (a, a + \delta)$, $f(x) \in (L - \varepsilon, L + \varepsilon)$.
\end{definition}

Alternatively,

\begin{definition}[Limit]
  If $\displaystyle\lim_{x \to a} f(x) = L$, then for any $\varepsilon > 0$, there exists $\delta > 0$ s.t. for any $| f (x) - L| < \varepsilon$ whenever $0 < |x - a| < \delta$.
\end{definition}

\subsubsection*{2.6 Limit Laws}

\begin{theorem}[Limit Laws]
  Let $c \in R$ be a constant and suppose the limits $\lim_{x \to a} f (x)$ and $\lim_{x \to a} g (x)$ exist. Then
  \begin{enumerate}[(i)]
  \item $\displaystyle\lim_{x \to a}( f (x) \pm g(x)) = \lim_{x \to a} f (x) \pm \lim_{x \to a} g(x)$
  \item $\displaystyle\lim_{x \to a}(c f (x)) = c \lim_{x \to a} f (x)$
  \item $\displaystyle\lim_{x \to a}( f (x)g(x)) = \lim_{x \to a} f (x) \lim_{x \to a} g(x)$
  \item $\displaystyle\lim_{x \to a} f (x) g(x) = \lim_{x \to a} f (x) \lim_{x \to a} g(x)$ , provided that $\displaystyle\lim_{x \to a} g(x) \neq 0$
  \item See (i).
  \item $\displaystyle\lim_{x \to a} x^n = (\lim_{x \to a} x)^n$
  \item $\displaystyle\lim_{x \to a} \sqrt{f(x)} = \sqrt{\lim_{x \to a} f(x)}$
  \item $\displaystyle\lim_{x \to a} \frac{a(x)b(x)}{c(x)b(x)} = \lim_{x \to a} \frac{a(x)}{c(x)}$
  \end{enumerate}
\end{theorem}

\begin{theorem}[Operations on infinity]
  For $x \in \Real$,
  \[
    \infty + x = \infty
  \]\[
    -\infty + x = -\infty
  \]\[
    x * \infty = \begin{cases}
      \infty & \text{if } x > 0 \\
      -\infty & \text{if } x < 0
    \end{cases}
  \]\[
    x * -\infty = \begin{cases}
      -\infty & \text{if } x > 0 \\
      \infty & \text{if } x < 0.
    \end{cases}
  \]\[
    \frac{x}{\pm\infty} = 0 
  \]
\end{theorem}

\begin{definition}[Indeterminate forms]
  The following forms are indeterminate and you cannot evaluate them.
  \[
    \frac{0}{0}, \frac{\pm\infty}{\pm\infty}, 0*\pm\infty, \infty - \infty
  \]
\end{definition}

\subsubsection*{Other theorems}

\begin{namedtheorem}[Composite Function Theorem]
  If $f$ is continuous at $L$ and $\displaystyle\lim_{x \to a}g(x) = L$, then $\displaystyle\lim_{x \to a} f(g(x) = f(lim_{x \to a} g(x))) = f(L)$
\end{namedtheorem}

\subsubsection*{2.12 Squeeze Theorem}

\begin{namedtheorem}[Squeeze Theorem]
  Let $f$ , $g$, and $h$ be defined for all $x \neq a$ over an open interval containing $a$. If
  $$
    f(x) \leq g(x) \leq h(x)
  $$
  for all $x \neq a$ in an open interval containing $a$ and
  $$
  \lim_{x \to a} f (x) = L = \lim_{x \to a} h(x)
  $$
  where $L \in \Real$, then $\lim_{x \to a} g(x) = L$.
\end{namedtheorem}

\subsection*{3 Continuity}

\begin{definition}[Continuity at a point]
  Function $f$ is continuous at point $a$ if $\displaystyle\lim_{x \to a} f (x) = f (a)$.
\end{definition}

\begin{definition}
  $f$ has a \textbf{removable discontinuity} if $\displaystyle\lim_{x \to a} f (x) = L \in \Real$ (in this case either $f (a)$ is undefined, or $f (a)$ is defined by $L \neq f (a)$).
\end{definition}
\begin{definition}
  $f$ has a \textbf{jump discontinuity} if $\displaystyle\lim_{x \to a^-} f (x) = L_1 \in \Real$ and $\displaystyle\lim_{x \to a^+} f (x) = L_2 \in \Real$ but $L1 \neq L2$.
\end{definition}
\begin{definition}
  $f$ has an \textbf{infinite discontinuity} at $a$ if $\displaystyle\lim_{x \to a^-} f (x) = \pm\infty$ or $\displaystyle\lim_{x \to a^+} f (x) = \pm\infty$
\end{definition}

\begin{namedtheorem}[Intermediate Value Theorem]
  If $f$ is continuous on $[a, b]$, then for any real number $L$ between $f (a)$ and $f (b)$ there exists at least one $c \in [a, b]$ such that $f (c) = L$. In other
  words, if $f$ is continuous on $[a, b]$, then the graph must cross the horizontal line $y = L$ at least once
  between the vertical lines $x = a$ and $x = b$.
\end{namedtheorem}

\begin{namedtheorem}[Aura Theorem]
  If $f(x)$ is continuous and $f(a)$ is positive, then there exists an open interval containing $a$ such that for all $x$ in the interval, $f(x)$ is positive.

  If $f(x)$ is continuous and $f(a)$ is negative, then there exists an open interval containing $a$ such that for all $x$ in the interval, $f(x)$ is negative.
\end{namedtheorem}

\begin{namedtheorem}[Bolzano's Theorem]
  Let $f$ be a continuous function defined on $[a, b]$. If $0$ is between $f (a)$ and $f (b)$, then there exists $x \in [a, b]$ such that $f (x) = 0$.
\end{namedtheorem}

\pagebreak

\subsection*{4 Derivatives}

The derivative is the instantaneous rate of change, and the slope of the tangent line to the point.

\begin{definition}[Derivative ($f'(a)$)]
  \[
    \frac{d}{da} f(a) = f'(a) = \lim_{x \to a} \frac{f(x)-f(a)}{x - a} = \lim_{h \to 0} \frac{f(a+h)-f(a)}{h}
  \]
\end{definition}

\begin{theorem}[Tangent line to a point]
  The equation of the tangent line to the point $(a, f(a))$ is
  \[
    y = f'(a)(x-a) + f(a)
  \]
\end{theorem}

\subsubsection*{Derivative Rules}

\begin{theorem}[Difference Rule]
  \[
    \frac{d}{dx}(f(x) - g(x)) = \frac{d}{dx}f(x) - \frac{d}{dx}g(x)
  \]
\end{theorem}

\begin{theorem}[Sum Rule]
  \[
    \frac{d}{dx}(f(x) + g(x)) = \frac{d}{dx}f(x) + \frac{d}{dx}g(x)
  \]
\end{theorem}

\begin{theorem}[Constant Multiple Rule]
  \[
    \frac{d}{dx}(cf(x)) = c\frac{d}{dx}f(x)
  \]
\end{theorem}

\begin{theorem}[Product Rule]
  \[\begin{aligned}
    \frac{d}{dx}(f(x)g(x)) &= &f'(x)g(x) + f(x)g'(x) \\
    \frac{d}{dx}(f(x)g(x)h(x)) &= &f'(x)g(x)h(x) + f(x)g'(x)h(x) \\
                               &+ &f(x)g(x)h'(x)
  \end{aligned}\]
  and so on.
\end{theorem}

\begin{theorem}[Quotient Rule]
  \[
    \frac{d}{dx}\frac{f(x)}{g(x)} = \frac{f'(x)g(x) - f(x)g'(x)}{(g(x))^2}
  \]
\end{theorem}

\begin{theorem}[Power Rule]
  \[
    \frac{d}{dx} x^n = nx^{n-1}
  \]
  for $n \in \Real$
\end{theorem}

\begin{theorem}[Chain Rule]
  \[
    \frac{d}{dx}f(g(x)) = f'(g(x))g'(x) \qquad \frac{dy}{dx} = \frac{dy}{db}\frac{db}{dx}
  \]
\end{theorem}

\begin{theorem}[Derivative of inverse functions]
  Let $x \in \Real$ and $f$ be a differentiable, one-to-one function at $x$. Then if $f'(x) \neq 0$, then
  \[
    (f^{-1})'(f(x)) = \frac{1}{f'(x)}
  \]
\end{theorem}

\begin{theorem}[Derivatives of exponentials and logs]
  \[\begin{aligned}
    \frac{d}{dx} e^x &= e^x &
    \frac{d}{dx} \ln x &= \frac{1}{x} \\
    \frac{d}{dx} a^x &= a^x \ln(a)&
    \frac{d}{dx} \log_a x &= \frac{1}{x\ln(a)}
  \end{aligned}\]
\end{theorem}

\begin{theorem}[Derivatives of trig functions]
  \[\begin{aligned}
    \sin'(x) &= \cos(x) &
    \cos'(x) &= -\sin(x) \\
    \sec'(x) &= \sec(x)\tan(x) &
    \csc'(x) &= -\csc(x)\cot(x) \\
    \tan'(x) &= \sec(x)^2 &
    \cot'(x) &= -\csc(x)^2
  \end{aligned}\]
  \[\begin{aligned}
    \arcsin'(x) &= \frac{1}{\sqrt{1-x^2}} &
    \arccos'(x) &= -\frac{1}{\sqrt{1-x^2}} \\
    \arcsec'(x) &= \frac{1}{|x|\sqrt{x^2-1}} &
    \arccsc'(x) &= -\frac{1}{|x|\sqrt{x^2-1}} \\
    \arctan'(x) &= \frac{1}{1+x^2} &
    \arccot'(x) &= -\frac{1}{1+x^2}
  \end{aligned}\]
\end{theorem}

\subsection*{5 Derivative Applications}

\subsubsection*{5.7 Mean Value Theorem}

\begin{theorem}[Mean Value Theorem]
  If the function $f$ is continuous on $[a, b]$ and differentiable on $(a, b)$, then there exists $c \in (a, b)$ s.t.
  \[
    f'(c) = \frac{f(b) - f(a)}{b - a} = \frac{\Delta f(x)}{\Delta x} \text{ on } [a, be]
  \]
\end{theorem}

\begin{theorem}[Some colloraries to the MVT]
  If $f(x)$ is differentiable on $I$, then:
  \begin{itemize}
    \item $f'(x) > 0$ for $x \in I$ $\iff$ $f(x)$ is strictly increasing for $x \in I$.
    \item $f'(x) \geq 0$ for $x \in I$ $\iff$ $f(x)$ is increasing or constant for $x \in I$.
    \item $f'(x) = 0$ for $x \in I$ $\iff$ $f(x)$ is constant for $x \in I$.
    \item $f'(x) \leq 0$ for $x \in I$ $\iff$ $f(x)$ is decreasing or constant for $x \in I$.
    \item $f'(x) < 0$ for $x \in I$ $\iff$ $f(x)$ is strictly decreasing for $x \in I$.
  \end{itemize}
\end{theorem}

\subsubsection*{Antiderivative}

\begin{definition}[Antiderivative]
  The antiderivative $F$ of a function $f$ is the function such that $F'(x) = f(x)$.
  \[
    F(x) = \int f(x) dx
  \]
\end{definition}

\subsubsection*{5.3, 5.10, 5.11}

\begin{definition}[Critical point of $f$]
  A number $c$ in the domain of $f$ where either $f'(c) \in {0, \text{DNE}}$
\end{definition}

\begin{definition}[Stationary point of $f$]
  A number $c$ in the domain of $f$ where either $f'(c) = 0$
\end{definition}

\begin{namedtheorem}[Fermat's Theorem]
  The local maxima and minima of $f$ are critical points of $f$.
\end{namedtheorem}

\begin{theorem}[Method to find absolute minima and maxima]
  Store the critical points of $f$ in the array $C$. Then, the absolute maximum is $\max {f(c) : c \in C}$ and the absolute minimum is $\min {f(c) : c \in C}$.
\end{theorem}

\begin{theorem}[First Derivative Test]
  If $f$ is continuous over $I$, and $c \in I$ is a critical point of $f$, and $f$ is differentiable over $I \setminus {c}$, then:

  \begin{itemize}
    \item If $f'(x)$ is decreasing at $c$, then $f(c)$ is a local max.
    \item If $f'(x)$ is increasing at $c$, then $f(c)$ is a local min.
    \item If $f'(x)$ has the same sign before and after $c$, then $f(c)$ is neither a local max nor a local min.
  \end{itemize}
\end{theorem}

\begin{definition}[Concavity]
  $f$ is concave up on $I$ if the tangent line to $f$ at each point in $I$ is lower than the graph of $f$.

  $f$ is concave down on $I$ if the tangent line to $f$ at each point in $I$ is higher than the graph of $f$.
\end{definition}

\begin{theorem}[Test for Concavity]
  If $f''(x) > 0$ for all $x \in I$, then $f$ is concave up on $I$.

  If $f''(x) < 0$ for all $x \in I$, then $f$ is concave down on $I$.
\end{theorem}

\begin{theorem}[Second Derivative Test]
  If $f''$ is continuous on an interval containing $c$, where $c$ is the $x$-value of a stationary point of $f$. Then,
  \begin{itemize}
    \item If $f''(c) > 0$, then $f(c)$ is a local max.
    \item If $f''(c) < 0$, then $f(c)$ is a local min.
  \end{itemize}
\end{theorem}